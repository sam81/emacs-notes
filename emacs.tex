\documentclass[a4paper,12pt]{report}

\usepackage{fontspec}
\usepackage[math-style=ISO]{unicode-math}
\setmainfont{Alegreya}
%\setmathfont{Libertinus Math}
\setmonofont{Ubuntu Mono}
\setsansfont{Kurier}

\usepackage{xcolor}
\definecolor{linkcol}{RGB}{0, 51, 153}
\usepackage{keystroke}
\usepackage{sectsty}
\allsectionsfont{\normalfont\sffamily\bfseries}

\usepackage{hyperref}
\hypersetup{pdftitle= emacs notes,
            pdfauthor=Samuele Carcagno,
            pdfpagemode=UseOutlines,
            pdfstartview=FitH,
            pdfkeywords={emacs, text, editing, programming, writing, text editor},
            colorlinks=true,
            linkcolor=linkcol,
            citecolor=linkcol,
            filecolor=linkcol,
            urlcolor=black  
} 

\title{Emacs Notes}
\author{Samuele Carcagno}
\begin{document}
\setlength{\parindent}{0pt}

\maketitle
\tableofcontents

\chapter{Emacs commands}





\section{Help}

\texttt{C-h} Help Menu \\

\section{Moving around}

First, please notice that the bar you can drag with your mouse for moving around a document is \textbf{on the left} rather than on the right. It got me some days to discover this \ldots \\

\texttt{C-a} Go to the beginning of the line \\

\texttt{C-b} Go to the end of the line \\

\texttt{M-a} Go to the beginning of a sentence \\

\texttt{M-e} Go to the end  of a sentence \\

\texttt{C-f} Move forward one character \\

\texttt{C-b} Move backward one character \\

\texttt{M-f} Move forward one word \\

\texttt{M-b} Move backward one word \\

\texttt{M-x goto-line} Press Enter, specify the line number and press Enter again. \\
Recent versions of Emacs have also
\texttt{M-g g} as a shortcut for \texttt{M-x goto-line} \\

\texttt{M-<} Move to beginning of file \\

\texttt{M->} Move to end of file \\

\section{Killin' 'n Yankin' (or Cutting and Pasting)}

\texttt{C-w} Kill \\

\texttt{M-w} Copy \\

\texttt{C-k} Kill line from cursor onwards \\

\texttt{C-d} or \texttt{canc} Kill character under cursor \\

\texttt{backspace} Kill character backwards \\

\texttt{M-d} Kill word forward \\

\texttt{M-canc} or \texttt{M-backspace} Kill word backwards \\

\texttt{C-spacebar} or \texttt{C-@} Set the mark \\

\section{Files and Buffers}

\texttt{C-x C-s} Save \\

\texttt{C-x C-f} Open file or create new file \\

\texttt{C-x k filename} Kill buffer (close file)\\

\texttt{M-x kill-some-buffers} Asks about each buffer, whether you want to kill it


\section{Windows}

\texttt{C-x 2} Split into two windows, one above and one below \\

\texttt{C-x 3} Split into two windows, side by side \\

\texttt{C-x o} Jump to the next window. This includes the mini-buffer's window, if that's active \\

\texttt{C-M-v} Make the other window scroll \\
\section{Searching \& Replacing}

\texttt{C-s} Search forward \\

\texttt{C-r} Search backwards \\

\texttt{C-s Ret C-w} Search forward only entire word \\

\texttt{C-r Ret C-w} Search backward only entire word \\

\texttt{M-x replace-string} Replace all occurrences of a string with another one, from the point where you are onward \\

\texttt{M-x query-replace} Ask before replacing \\

\section{Repeating a Character or Command}

\texttt{M-n c} Repeat the character `\texttt{c}' $n$ times\\ 

\section{Using Emacs in a Terminal}

\texttt{emacs -nw} Start Emacs in terminal mode \\

\texttt{F10} Activate menu bar \\

\section{Fonts}
One way to change font is to add an entry into the \verb+~/.Xresources+ file, for example:
\begin{verbatim}
Emacs.font: Monospace-10
\end{verbatim}
then run on a terminal
\begin{verbatim}
xrdb -merge ~/.Xresources
\end{verbatim}
to change the setting for the current X session.


\chapter{Spell Checking: Ispell and Aspell}
\label{chap:spellchecking}

If you have \texttt{Ispell} installed, you can use it for spell-checking with \texttt{Emacs}. If you want to use \texttt{Aspell} put this in your \texttt{.emacs}
\begin{verbatim}
  (setq-default ispell-program-name "aspell")
\end{verbatim}

To start spell-checking:\\

\texttt{M-x ispell} Check the current buffer (all the file), if Transient Mark Mode is on, it spell-checks the active region\\

\texttt{M-x ispell-region} Check region \\

\texttt{M-x ispell-word} or \texttt{M-\$} Check word \\

\texttt{M-x flyspell-mode} Highlight all misspelled words \\

\texttt{M-x ispell-change-dictionary} Change the dictionary, to use for example the italian, british or other dictionaries. If you know the name of the dictionary, you can directly input it in the mini-buffer when prompted, otherwise press \verb+SPC+ for a list of available dictionaries. Remember that you might have to first install a given aspell dictionary to make it available.\\



\chapter{Auctex}

\section{Execute Command}

\texttt{C-c C-c} plus command to execute \\

\section{Fonts}

\texttt{C-c C-f C-t} Typewriter \\

\texttt{C-c C-f C-e} Emphasise \\

\texttt{C-c C-f C-b} Boldface \\


\section{Sectioning}
\label{sec:sec}

\texttt{C-c C-s} Insert sectioning command\\

\section{Environments}

\texttt{C-c C-e verbatim} Verbatim \\

\texttt{C-c C-j} Insert Item\\


\section{Multifile/Parsing}
\label{sec:multi}

\texttt{C-c \_} Set master file\\


\section{Checking \LaTeX\ Syntax}

\texttt{C-c C-c Check} Check the file\\

\texttt{C-x `} or \texttt{C-x next-error} to jump to the errors in sequence \\


\chapter{Aspell}
\label{chap:aspell}

If you want to spell-check using aspell directly from the command line rather than from within Emacs, here's some good tips. To start spell-checking
\begin{verbatim}
  $ aspell -c filename.txt
\end{verbatim}
the \texttt{add} option is very useful, it adds the current word to a personal dictionary, so next time aspell encounter that word, it won't highlight it as a possible error. The personal dictionary is a simple text file that is automatically created in your home directory. In my case, the file is named \verb+.aspell.en.pws+, you should substitute \verb+en+ with the code for the dictionary language you're using (e.g. \verb+it+ for Italian).

\chapter{Emacs packages}

To add a package repository (e.g.\ melpa-stable), add the following to your \verb+.emacs+:
\begin{verbatim}
(package-initialize)

(add-to-list 'package-archives
             '("melpa-stable" . "https://stable.melpa.org/packages/") t)
\end{verbatim}

Updating the package repository:
\begin{verbatim}
M-x package-refresh-contents
\end{verbatim}

Listing available packages:
\begin{verbatim}
M-x package-list-packages
\end{verbatim}

\end{document}

%%% Local Variables:
%%% mode: latex
%%% TeX-engine: luatex
%%% TeX-master: "emacs"
%%% End: 